\documentclass[a5paper, twocolumn]{tbook}
\usepackage[expert, deluxe]{otf}
\usepackage{otf}
\usepackage{url}
\usepackage{furikana}
\usepackage{type1cm}
\usepackage[final]{aozora}
%\usepackage{chouchou}
\usepackage{here}
\usepackage{aozora}
\def\rubykatuji{\rubyfamily\tiny}
\def\rubykatuji{\tiny}%for UTF package
\renewenvironment{teihon}{\comment}{\endcomment}
%\usepackage{okumacro}
\title{オツベルと象}
\author{宮沢賢治}
\date{}
\begin{document}
\maketitle
       ……ある\ruby{牛飼}{うしか}いがものがたる
 
\endnote{3字下げ}第一日曜\endnote{「第一日曜」は中見出し}
 
 オツベルときたら大したもんだ。\Ruby{稲扱}{いねこき}器械の六台も\ruby{据}{す}えつけて、のんのんのんのんのんのんと、大そろしない音をたててやっている。
 十六人の\ruby{百姓}{ひゃくしょう}どもが、顔をまるっきりまっ赤にして足で\ruby{踏}{ふ}んで器械をまわし、小山のように積まれた稲を片っぱしから\ruby{扱}{こ}いて行く。\ruby{藁}{わら}はどんどんうしろの方へ投げられて、また新らしい山になる。そこらは、\ruby{籾}{もみ}や藁から\ruby{発}{た}ったこまかな\ruby{塵}{ちり}で、変にぼうっと黄いろになり、まるで\ruby{沙漠}{さばく}のけむりのようだ。
 そのうすくらい仕事場を、オツベルは、大きな\ruby{琥珀}{こはく}のパイプをくわえ、\ruby{吹殻}{ふきがら}を藁に落さないよう、\ruby{眼}{め}を細くして気をつけながら、両手を背中に組みあわせて、ぶらぶら\ruby{往}{い}ったり来たりする。
 小屋はずいぶん\ruby{頑丈}{がんじょう}で、学校ぐらいもあるのだが、何せ新式稲扱器械が、六台もそろってまわってるから、のんのんのんのんふるうのだ。中にはいるとそのために、すっかり腹が\ruby{空}{す}くほどだ。そしてじっさいオツベルは、そいつで上手に腹をへらし、ひるめしどきには、六寸ぐらいのビフテキだの、\ruby{雑巾}{ぞうきん}ほどあるオムレツの、ほくほくしたのをたべるのだ。
 とにかく、そうして、のんのんのんのんやっていた。
 そしたらそこへどういうわけか、その、白象がやって来た。白い象だぜ、ペンキを\ruby{塗}{ぬ}ったのでないぜ。どういうわけで来たかって?{}そいつは象のことだから、たぶんぶらっと森を出て、ただなにとなく来たのだろう。
 そいつが小屋の入口に、ゆっくり顔を出したとき、百姓どもはぎょっとした。なぜぎょっとした?{}よくきくねえ、何をしだすか知れないじゃないか。かかり合っては大へんだから、どいつもみな、いっしょうけんめい、じぶんの稲を扱いていた。
 ところがそのときオツベルは、ならんだ器械のうしろの方で、ポケットに手を入れながら、ちらっと\ruby{鋭}{するど}く象を見た。それからすばやく下を向き、何でもないというふうで、いままでどおり往ったり来たりしていたもんだ。
 するとこんどは白象が、\RUBY{片脚}{かたあし}\RUBY{床}{ゆか}にあげたのだ。百姓どもはぎょっとした。それでも仕事が\ruby{忙}{いそが}しいし、かかり合ってはひどいから、そっちを見ずに、やっぱり稲を扱いていた。
 オツベルは\ruby{奥}{おく}のうすくらいところで両手をポケットから出して、も一度ちらっと象を見た。それからいかにも\ruby{退屈}{たいくつ}そうに、わざと大きなあくびをして、両手を頭のうしろに組んで、行ったり来たりやっていた。ところが象が\ruby{威勢}{いせい}よく、\Ruby{前肢}{まえあし}二つつきだして、小屋にあがって来ようとする。百姓どもはぎくっとし、オツベルもすこしぎょっとして、大きな琥珀のパイプから、ふっとけむりをはきだした。それでもやっぱりしらないふうで、ゆっくりそこらをあるいていた。
 そしたらとうとう、象がのこのこ上って来た。そして器械の前のとこを、\ruby{呑気}{のんき}にあるきはじめたのだ。
 ところが何せ、器械はひどく\ruby{廻}{まわ}っていて、\ruby{籾}{もみ}は夕立か\ruby{霰}{あられ}のように、パチパチ象にあたるのだ。象はいかにもうるさいらしく、小さなその眼を細めていたが、またよく見ると、たしかに少しわらっていた。
 オツベルはやっと\ruby{覚悟}{かくご}をきめて、\Ruby{稲扱}{いねこき}器械の前に出て、象に話をしようとしたが、そのとき象が、とてもきれいな、\ruby{鶯}{うぐいす}みたいないい声で、こんな文句を\ruby{云}{い}ったのだ。
「ああ、だめだ。あんまりせわしく、砂がわたしの歯にあたる。」
 まったく籾は、パチパチパチパチ歯にあたり、またまっ白な頭や首にぶっつかる。
 さあ、オツベルは\ruby{命懸}{いのちが}けだ。パイプを右手にもち直し、度胸を据えて\ruby{斯}{こ}う云った。
「どうだい、\ruby{此処}{ここ}は\ruby{面白}{おもしろ}いかい。」
「面白いねえ。」象がからだを\ruby{斜}{なな}めにして、眼を細くして返事した。
「ずうっとこっちに居たらどうだい。」
 百姓どもははっとして、息を殺して象を見た。オツベルは云ってしまってから、にわかにがたがた\ruby{顫}{ふる}え出す。ところが象はけろりとして
「居てもいいよ。」と答えたもんだ。
「そうか。それではそうしよう。そういうことにしようじゃないか。」オツベルが顔をくしゃくしゃにして、まっ赤になって\ruby{悦}{よろこ}びながらそう云った。
 どうだ、そうしてこの象は、もうオツベルの財産だ。いまに見たまえ、オツベルは、あの白象を、はたらかせるか、サーカス団に売りとばすか、どっちにしても万円以上もうけるぜ。
 
\endnote{3字下げ}第二日曜\endnote{「第二日曜」は中見出し}
 
 オツベルときたら大したもんだ。それにこの前稲扱小屋で、うまく自分のものにした、象もじっさい大したもんだ。力も二十馬力もある。第一みかけがまっ白で、\ruby{牙}{きば}はぜんたいきれいな\ruby{象牙}{ぞうげ}でできている。皮も全体、立派で\ruby{丈夫}{じょうぶ}な象皮なのだ。そしてずいぶんはたらくもんだ。けれどもそんなに\ruby{稼}{かせ}ぐのも、やっぱり主人が\ruby{偉}{えら}いのだ。
「おい、お前は時計は\ruby{要}{い}らないか。」丸太で建てたその象小屋の前に来て、オツベルは琥珀のパイプをくわえ、顔をしかめて斯う\ruby{訊}{き}いた。
「ぼくは時計は要らないよ。」象がわらって返事した。
「まあ持って見ろ、いいもんだ。」斯う言いながらオツベルは、ブリキでこさえた大きな時計を、象の首からぶらさげた。
「なかなかいいね。」象も云う。
「\ruby{鎖}{くさり}もなくちゃだめだろう。」オツベルときたら、百キロもある鎖をさ、その前肢にくっつけた。
「うん、なかなか鎖はいいね。」三あし歩いて象がいう。
「\ruby{靴}{くつ}をはいたらどうだろう。」
「ぼくは靴などはかないよ。」
「まあはいてみろ、いいもんだ。」オツベルは顔をしかめながら、赤い張子の大きな靴を、象のうしろのかかとにはめた。
「なかなかいいね。」象も云う。
「靴に\ruby{飾}{かざ}りをつけなくちゃ。」オツベルはもう大急ぎで、四百キロある分銅を靴の上から、\ruby{穿}{は}め込んだ。
「うん、なかなかいいね。」象は二あし歩いてみて、さもうれしそうにそう云った。
 次の日、ブリキの大きな時計と、やくざな紙の靴とはやぶけ、象は鎖と分銅だけで、大よろこびであるいて\ruby{居}{お}った。
「済まないが税金も高いから、今日はすこうし、川から水を\ruby{汲}{く}んでくれ。」オツベルは両手をうしろで組んで、顔をしかめて象に云う。
「ああ、ぼく水を汲んで来よう。もう何ばいでも汲んでやるよ。」
 象は眼を細くしてよろこんで、そのひるすぎに五十だけ、川から水を汲んで来た。そして菜っ葉の畑にかけた。
 夕方象は小屋に居て、十\Ruby{把}{ぱ}の\ruby{藁}{わら}をたべながら、西の三日の月を見て、
「ああ、\ruby{稼}{かせ}ぐのは\ruby{愉快}{ゆかい}だねえ、さっぱりするねえ」と云っていた。
「済まないが税金がまたあがる。今日は少うし森から、たきぎを運んでくれ」オツベルは\ruby{房}{ふさ}のついた赤い\ruby{帽子}{ぼうし}をかぶり、両手をかくしにつっ込んで、次の日象にそう言った。
「ああ、ぼくたきぎを持って来よう。いい天気だねえ。ぼくはぜんたい森へ行くのは大すきなんだ」象はわらってこう言った。
 オツベルは少しぎょっとして、パイプを手からあぶなく落としそうにしたがもうあのときは、象がいかにも愉快なふうで、ゆっくりあるきだしたので、また安心してパイプをくわえ、小さな\ruby{咳}{せき}を一つして、百姓どもの仕事の方を見に行った。
 そのひるすぎの半日に、象は九百把たきぎを運び、眼を細くしてよろこんだ。
 晩方象は小屋に居て、八把の藁をたべながら、西の四日の月を見て
「ああ、せいせいした。サンタマリア」と\ruby{斯}{こ}うひとりごとしたそうだ。
 その次の日だ、
「済まないが、税金が五倍になった、今日は少うし\ruby{鍛冶場}{かじば}へ行って、炭火を\ruby{吹}{ふ}いてくれないか」
「ああ、吹いてやろう。本気でやったら、ぼく、もう、息で、石もなげとばせるよ」
 オツベルはまたどきっとしたが、気を落ち付けてわらっていた。
 象はのそのそ鍛冶場へ行って、べたんと肢を折って\ruby{座}{すわ}り、ふいごの代りに半日炭を吹いたのだ。
 その晩、象は象小屋で、七\Ruby{把}{わ}の藁をたべながら、空の五日の月を見て
「ああ、つかれたな、うれしいな、サンタマリア」と斯う言った。
 どうだ、そうして次の日から、象は朝からかせぐのだ。藁も昨日はただ五把だ。よくまあ、五把の藁などで、あんな力がでるもんだ。
 じっさい象はけいざいだよ。それというのもオツベルが、頭がよくてえらいためだ。オツベルときたら大したもんさ。
 
\endnote{3字下げ}第五日曜\endnote{「第五日曜」は中見出し}
 
 オツベルかね、そのオツベルは、おれも云おうとしてたんだが、居なくなったよ。
 まあ落ちついてききたまえ。前にはなしたあの象を、オツベルはすこしひどくし過ぎた。しかたがだんだんひどくなったから、象がなかなか笑わなくなった。時には赤い\ruby{竜}{りゅう}の眼をして、じっとこんなにオツベルを見おろすようになってきた。
 ある晩象は象小屋で、三把の藁をたべながら、十日の月を\ruby{仰}{あお}ぎ見て、
「苦しいです。サンタマリア。」と云ったということだ。
 こいつを聞いたオツベルは、ことごと象につらくした。
 ある晩、象は象小屋で、ふらふら\ruby{倒}{たお}れて地べたに座り、藁もたべずに、十一日の月を見て、
「もう、さようなら、サンタマリア。」と斯う言った。
「おや、何だって?{}さよならだ?」月が\ruby{俄}{にわ}かに象に\ruby{訊}{き}く。
「ええ、さよならです。サンタマリア。」
「何だい、なりばかり大きくて、からっきし\ruby{意気地}{いくじ}のないやつだなあ。仲間へ手紙を書いたらいいや。」月がわらって斯う云った。
「お筆も紙もありませんよう。」象は細ういきれいな声で、しくしくしくしく泣き出した。
「そら、これでしょう。」すぐ眼の前で、\ruby{可愛}{かあい}い子どもの声がした。象が頭を上げて見ると、赤い着物の童子が立って、\ruby{硯}{すずり}と紙を\ruby{捧}{ささ}げていた。象は早速手紙を書いた。
「ぼくはずいぶん眼にあっている。みんなで出て来て助けてくれ。」
 童子はすぐに手紙をもって、林の方へあるいて行った。
 \ruby{赤衣}{せきい}の童子が、そうして山に着いたのは、ちょうどひるめしごろだった。このとき山の象どもは、\ruby{沙羅樹}{さらじゅ}の下のくらがりで、\ruby{碁}{ご}などをやっていたのだが、額をあつめてこれを見た。
「ぼくはずいぶん眼にあっている。みんなで出てきて助けてくれ。」
 象は一せいに立ちあがり、まっ黒になって\ruby{吠}{ほ}えだした。
「オツベルをやっつけよう」議長の象が高く\ruby{叫}{さけ}ぶと、
「おう、でかけよう。グララアガア、グララアガア。」みんながいちどに呼応する。
 さあ、もうみんな、\ruby{嵐}{あらし}のように林の中をなきぬけて、グララアガア、グララアガア、野原の方へとんで行く。どいつもみんなきちがいだ。小さな木などは根こぎになり、\ruby{藪}{やぶ}や何かもめちゃめちゃだ。グワア グワア グワア グワア、花火みたいに野原の中へ飛び出した。それから、何の、走って、走って、とうとう向うの青くかすんだ野原のはてに、オツベルの\ruby{邸}{やしき}の黄いろな屋根を\ruby{見附}{みつ}けると、象はいちどに\ruby{噴火}{ふんか}した。
 グララアガア、グララアガア。その時はちょうど一時半、オツベルは皮の\ruby{寝台}{しんだい}の上でひるねのさかりで、\ruby{烏}{からす}の\ruby{夢}{ゆめ}を見ていたもんだ。あまり大きな音なので、オツベルの家の百姓どもが、門から少し外へ出て、小手をかざして向うを見た。林のような象だろう。汽車より早くやってくる。さあ、まるっきり、血の気も失せてかけ\ruby{込}{こ}んで、
「\ruby{旦那}{だんな}あ、象です。押し寄せやした。旦那あ、象です。」と声をかぎりに叫んだもんだ。
 ところがオツベルはやっぱりえらい。眼をぱっちりとあいたときは、もう何もかもわかっていた。
「おい、象のやつは小屋にいるのか。居る?{}居る?{}居るのか。よし、戸をしめろ。戸をしめるんだよ。早く象小屋の戸をしめるんだ。ようし、早く丸太を持って来い。とじこめちまえ、\ruby{畜生}{ちくしょう}めじたばたしやがるな、丸太をそこへしばりつけろ。何ができるもんか。わざと力を減らしてあるんだ。ようし、もう五六本持って来い。さあ、大丈夫だ。大丈夫だとも。あわてるなったら。おい、みんな、こんどは門だ。門をしめろ。かんぬきをかえ。つっぱり。つっぱり。そうだ。おい、みんな心配するなったら。しっかりしろよ。」オツベルはもう\ruby{支度}{したく}ができて、ラッパみたいないい声で、百姓どもをはげました。ところがどうして、百姓どもは気が気じゃない。こんな主人に巻き\ruby{添}{ぞ}いなんぞ食いたくないから、みんなタオルやはんけちや、よごれたような白いようなものを、ぐるぐる\ruby{腕}{うで}に巻きつける。降参をするしるしなのだ。
 オツベルはいよいよやっきとなって、そこらあたりをかけまわる。オツベルの犬も気が立って、火のつくように\ruby{吠}{ほ}えながら、やしきの中をはせまわる。
 間もなく地面はぐらぐらとゆられ、そこらはばしゃばしゃくらくなり、象はやしきをとりまいた。グララアガア、グララアガア、その\ruby{恐}{おそ}ろしいさわぎの中から、
「今助けるから安心しろよ。」やさしい声もきこえてくる。
「ありがとう。よく来てくれて、ほんとに\ruby{僕}{ぼく}はうれしいよ。」象小屋からも声がする。さあ、そうすると、まわりの象は、一そうひどく、グララアガア、グララアガア、\ruby{塀}{へい}のまわりをぐるぐる走っているらしく、度々中から、\ruby{怒}{おこ}ってふりまわす鼻も見える。けれども塀はセメントで、中には鉄も入っているから、なかなか象もこわせない。塀の中にはオツベルが、たった一人で叫んでいる。百姓どもは眼もくらみ、そこらをうろうろするだけだ。そのうち外の象どもは、仲間のからだを台にして、いよいよ塀を\ruby{越}{こ}しかかる。だんだんにゅうと顔を出す。その\ruby{皺}{しわ}くちゃで灰いろの、大きな顔を見あげたとき、オツベルの犬は気絶した。さあ、オツベルは\ruby{射}{う}ちだした。六連発のピストルさ。ドーン、グララアガア、ドーン、グララアガア、ドーン、グララアガア、ところが\ruby{弾丸}{たま}は通らない。\ruby{牙}{きば}にあたればはねかえる。一\Ruby{疋}{ぴき}なぞは\ruby{斯}{こ}う言った。
「なかなかこいつはうるさいねえ。ぱちぱち顔へあたるんだ。」
 オツベルはいつかどこかで、こんな文句をきいたようだと思いながら、ケースを帯からつめかえた。そのうち、象の片脚が、塀からこっちへはみ出した。それからも一つはみ出した。五匹の象が一ぺんに、塀からどっと落ちて来た。オツベルはケースを握ったまま、もうくしゃくしゃに\ruby{潰}{つぶ}れていた。早くも門があいていて、グララアガア、グララアガア、象がどしどしなだれ込む。
「\ruby{牢}{ろう}はどこだ。」みんなは小屋に押し寄せる。丸太なんぞは、マッチのようにへし折られ、あの白象は大へん\ruby{瘠}{や}せて小屋を出た。
「まあ、よかったねやせたねえ。」みんなはしずかにそばにより、鎖と銅をはずしてやった。
「ああ、ありがとう。ほんとにぼくは助かったよ。」白象はさびしくわらってそう云った。
 おや〔一字不明〕、川へはいっちゃいけないったら。
 
 
 
\theendnotes
\begin{teihon}
\clearpage\thispagestyle{empty}
\begin{minipage}<y>[t]{\textheight}
\vspace{0.5\baselineskip}
\scriptsize
底本:「新編 銀河鉄道の夜」新潮文庫、新潮社
   1989(平成元)年6月15日発行
底本の親本:「新修宮沢賢治全集 第十三巻」筑摩書房
   1980(昭和55)年3月
※「〔一字不明〕」は、底本編集時の注記です。
入力:r.sawai
校正:篠宮康彰
1999年2月6日公開
2011年2月14日修正
青空文庫作成ファイル:
このファイルは、インターネットの図書館、青空文庫(http://www.aozora.gr.jp)で作られました。入力、校正、制作にあたったのは、ボランティアの皆さんです。

\vspace{0.5\baselineskip}
お断り:このPDFファイルは、青空パッケージ(http://psitau.kitunebi.com/aozora.html)を使って自動的に作成されたものです。従って、著作の底本通りではなく、制作者は、WYSIWYG(見たとおりの形)を保証するものではありません。不具合は、http://www.aozora.jp/blog2/2008/06/16/62.html\ までコメントの形で、ご報告ください。


\end{minipage}
\end{teihon}
\end{document}
